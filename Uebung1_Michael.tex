\input{src/header}										
\newcommand{\dozent}{Wolfgang Mulzer, Katharina Klost}					% <-- Names des Dozenten eintragen
\newcommand{\tutor}{Tobias Gleißner}						% <-- Name eurer Tutoriun eintragen
\newcommand{\tutoriumNo}{02}				% <-- Nummer im KVV nachschauen
\newcommand{\ubungNo}{01}									% <-- Nummer des Übungszettels
\newcommand{\veranstaltung}{Algorithmen, Datenstrukturen und Datenabstraktion}	% <-- Name der Lehrveranstaltung eintragen
\newcommand{\semester}{Semester}						% <-- z.B. SoSo 17, WiSe 17/18
\newcommand{\studenten}{Nicolas Benjamin, Michael Wernitz}			% <-- Hier eure Namen eintragen
\newcommand{\aufgNo}{3}	

\usepackage{amsmath}
\usepackage{mathtools}
\usepackage{letltxmacro}

% /////////////////////// BEGIN DOKUMENT /////////////////////////
\begin{document}							
% ////////////// Notenspiegel und Logo //////////////
\vspace*{-15ex}							% rückt Logo an den oberen Seitenrand
\makebox[\dimexpr\textwidth+1cm][r]{		%rechtsbündig und geht rechts 1cm über Layout hinaus
	\begin{minipage}{0.5\linewidth}
	\newcounter{AufgNo}
	\setcounter{AufgNo}{\aufgNo}
	\stepcounter{AufgNo}					% AufgNo++
	\newcounter{zahl}
	\def\and{&\xspace}
	\renewcommand{\arraystretch}{1.3}\setlength{\tabcolsep}{1em}
	\begin{tabular}{*{\value{AufgNo}}{|c} |}
		\hline
		\setcounter{zahl}{1}
		\whiledo{\value{zahl} < \value{AufgNo}}{%\AufgNo 
			\thezahl\and\stepcounter{zahl}%
		} $\sum$ \\ \hline
		\setcounter{zahl}{1}
		\whiledo{\value{zahl} < \value{AufgNo}}{%\AufgNo 
			\and\stepcounter{zahl}%
		} \\ \hline
	\end{tabular}

	%\forloop{ct}{1}{\value{ct} < 5}{}
	\end{minipage}
	\hfill
	\begin{minipage}{0.5\linewidth}
	\hfill \ifnum\aufgNo<9%				blendet ab 9 Aufgaben das Logo aus
	\includegraphics[width=0.8\textwidth]{src/fu_logo} % fügt FU-Logo ein
	\fi
	\end{minipage}
}
% ////////////// Daten //////////////
\begin{center}
{\large \dozent}\par
{\huge \veranstaltung, \semester}\par
{\large TutorIn: \tutor, Tutorium \tutoriumNo}\par
{\Large Übung \ubungNo}\par 
{\large \studenten}\par
\today
\end{center}
\vspace{-3ex}							% Abstand
\rule{\linewidth}{0.8pt}				% horizontale Linie

\section{ Rekursionsgleichung \hfill}
{\itshape Lösen Sie die folgenden Rekursionsgleichungen. Beweisen Sie jeweils die Richtigkeit Ihrer Lösung (z.B.: durch vollständige Induktion)}

\begin{flushleft}
	(a)
\end{flushleft}

\begin{center}
$T(1) = 0; T(n) = T(\lfloor{\frac{n}{2}}\rfloor) + T(\lceil{\frac{n}{2}}\rceil) + n$, für n > 1 
\end{center}



\begin{itemize}
\item  Ergebnisse durch Einsetzen von Zweierpotenzen von n\\
	$T(2) = T(1) + T(1) + 2 = 2$ \\
    $T(4) = T(2) + T(2) + 4 = 8$ \\
    $T(8) = T(4) + T(4) + 8 = 24$ \\
    $T(16) = T(8) + T(8) + 16 = 64$ \\
\end{itemize}
 
Daraus erkennt man die geschlossene Formel: $a_{n} = n * log_{2}n$

\begin{proof}
        Es folgt der dazugehörige Beweis durch vollständige Induktion 
        \par
        \begingroup
        \leftskip = 1,35cm
        \noindent
			Dazu betrachten wir n als $n=2^k$\\
			Es wird daher über k induziert
		\par
        \endgroup
        

	\begin{itemize}
	\item Induktionsanfang mit $k=1$, also $2^1$
		$T(2) = T(1) + T(1) + 2$\\
		$T(2) = 2$\\
     	$a_2 = 2 * log(2)$\\
    	$a_2 = 2$\\
     	$=> T(2) = a_2$
        
    \item Induktionsvorraussetzung: $T(n) = a_{n}$\\
    	wobei $T(2^k) = T(\lfloor{\frac{2^{k-1}}{2}}\rfloor) + T(\lceil{\frac{2^{k-1}}{2}}\rceil) + 2^k$\\
        \\
    	und $a_{2^k} = 2^k * log_{2}(2^k)$
    \item Induktionsbehauptung: $a_2^{k+1} = T(2^{k+1})$ 
    	wir gehen dabei davon aus, dass $T(2^{k+1})$ bereits bewiesen wurde, da $T(2^k)$ bzw. $T(n)$ in der Aufgabenstellung gegeben ist.   
    \item Induktionsschritt $k \mapsto k+1$ \\
       
       $T(2^{k+1}) = 2 * T(2^k) + 2^{k+1}$
       \par
       \begingroup
       \leftskip = 1,35cm 
       \noindent
       $ \overset{IV}{=} 2 * 2^k * log_{2}(2^k) + 2^{k+1}$ \\
       \\
       $ = 2^{k+1} * log_{2}(2^k) + 2^{k+1}$ \\
       \\
       $ = 2^{k+1} * (log_{2}(2^k) + 1)$ \\
       \\
       $ = 2^{k+1} * (log_{2}(2^k) + log_{2}2)$\\
       \\
       $ = 2^{k+1} * (log_{2}(2^k * 2^1)$ \\
       \\
       $ = 2^{k+1} * log_{2}(2^{k+1}) \Leftrightarrow a_{k+1}$ 
       \par
       \endgroup  
      
   	 \end{itemize}
	\end{proof}
    
\begin{flushleft}
	(b)
\end{flushleft}
\begin{center}
$S(1) = 1; S(n) = \displaystyle\sum_{i=1}^{n-1}i * S(i)$, für n > 1 
\end{center}
	\begin{itemize}
	
       $S(1) = 1 = 1$ \\
       $S(2) = 1 * S(1) = 1$ \\
       $S(3) = S(1) + 2*S(2) = 3$ \\
       $S(4) = S(1) + 2*S(2) + 3*S(3) = 12$ \\
       $S(5) = S(1) + 2*S(2) + 3*S(3) + 4*S(4)$
       \\
       Daraus ist zu erkennen: 
       $S(1) = 1$ \\
       $S(2) = 1$ \\
       $S(3) = 6 = 3 * 2$ \\
       $S(4) = 24 = 4 * 3 * 2 $ \\
       $S(5) = 120 = 5 * 4 * 3 * 2$ \\
       \\
       folglich kann allgemein gesagt werden: \\
       \[ a(n)= 
       		\begin{cases}
       			1 &\quad 0 < n}\leq2  \\     			
       			\frac{n!}{2} &\quad \text{sonst}
       		\end{cases}
       \]
       
       \begin{proof}	
	   \[       
       S(1) = 1, \quad \\
       S(n) = \displaystyle\sum_{i=1}^{n-1}i * S(i)  \quad \text{für n>1} \\
       \\
       \]
       \[
        a(n) = 
       		\begin{cases}
       			1 &\quad 0 < n\leq2  \\     			
       			\frac{n!}{2} &\quad \text{sonst}
       		\end{cases}
       \]
       \newpage
       
       Induktionsanfang: \\
       n=1 \\
       $a(1) = S(1)$ \\
       n = 2\\
       $a(2) = S(2)$ \\
       gewähltes n > 2, also sei n=3\\
       $S(3) = 3 = a(n)$ \\
       \\
       Induktionvorraussetzung: S(n) = a(n)\\
       \\
       Induktionsbehauptung: \\
       S(n+1) = a(n+1), wobei vorausgesetzt ist, dass S(n)bereits per Induktion bewiesen wurde\\
	   \\       
       Induktionsschritt: \quad $n \rightarrow n+1$ \\
       \\
       
	   \begin{equation}
		\begin{aligned}
			S(n+1)&=\left\displaystyle\sum_{i=1}^{n} i * S(i) \\
				  &= n * S(n) + \displaystyle\sum_{i=1}^{n-1} i * S(i) \\
				  &= n * S(n) + S(n) \notag \\
				  &= S(n) * (n+1) \\
				  &\overset{IV}{=} \frac{n!}{2} * (n+1) \\
				  &= \frac{(n+1)!}{2} = a(n+1) \right
		\end{aligned}			  	   
	   \end{equation}	          
       \end{proof}
   \end{itemize}    
	
	         
\newpage


\section{Die Ungleichung vom arithmetischen und geometrischen Mittel}


Seien $x_1, x_2, \ldots, x_n \geq 0$. Beweisen Sie
die Ungleichung
\[
P(n): x_1 x_2 \ldots x_n
\leq\left(\frac{x_1+\dots+x_n}{n}\right)^n,
\]
indem Sie folgende Teilschritte bearbeiten:
\begin{enumerate}
  \item $P(2)$ ist eine wahre Aussage.

     \emph{Hinweis}: $(x_1 - x_2)^2 \geq 0$.
  \item
     Wenn $P(n)$ gilt, dann gilt auch $P(n-1)$.

     \emph{Hinweis}: Wie muss man $x_n$ w\"ahlen, damit 
     sich der Wert der rechten Klammer nicht \"andert, wenn
     $x_1, \ldots, x_{n-1}$ schon feststehen?

   \item Aus $P(2)$ und $P(n)$ folgt $P(2n)$.
   \item Folgern Sie nun die Ungleichung. 
\end{enumerate}
\\
Lösungen: \\
\begin{itemize}
	\item 2a)\\
		$P(2)\quad \text{ist wahre Aussage}$ \\
		\begin{proof}
			\begin{equation}
				\begin{aligned}
					\left			
					 x_1 * x_2 &\leq (\frac{x_1+x_2}{2})^2	\\
					 x_1 * x_2 &\leq \frac{(x_1+x_2)^2}{4}		\\
					 4*x_1*x_2 &\leq (x_1 + x_2)^2 \notag	\\
					 4*x_1*x_2 &\leq {x_1}^2 + 2*x_1*x_2 + {x_2}^2 \\
					 0 &\leq {x_1}^2 - 2*x_1*x_2 + {x_2}^2 \\
					 0 &\leq (x_1 - x_2)^2 \right
				\end{aligned}
			\end{equation}
		\end{proof}
	\item 2b\\
		$P(n) \leftrightarrow P(n-1)$ \\
		\begin{proof} \\
			Es gilt: \\
			\begin{equation}
				\begin{aligned}
					x_1*x_2*x_3 \dots x_n &\leq \left(\frac{x_1+x_2+ \dots +x_n}{n}\right)^n 
					\Leftrightarrow \sqrt[n]{x_1*x_2*x_3 \dots x_n} &\leq \left(\frac{x_1+x_2+ \dots +x_n}{n}\right) \notag \\
				\end{aligned}
			\end{equation}
			Sei $ x_n = \sqrt[n-1]{x_1x_2 \dots x_{n-1}}$ \\
			Nach Einsetzen gilt: \\
			\begin{equation}
				\begin{aligned}
					\sqrt[n]{x_1x_2 \dots x_{n-1}*\sqrt[n-1]{x_1x_2 \dots x_{n-1}}} &\overset{?}{=} \sqrt[n-1]{x_1x_2 \dots x_{n-1}} \notag \\
				\sqrt[n]{x_1x_2 \dots x_{n-1}}*\sqrt[n]{\sqrt[n-1]{x_1x_2 \dots x_{n-1}}} &\Leftrightarrow \left(x_1x_2 \dots x_{n-1}\right)^{\frac{1}{n}} * \left(\sqrt[n-1]{x_1x_2 \dots x_{n-1}} \right)^{\frac{1}{n}} \\
				&\Leftrightarrow \left(x_1x_2 \dots x_{n-1}\right)^{\frac{1}{n}} * \left(\left(x_1x_2 \dots x_{n-1}\right)^{\frac{1}{n-1}}\right)^{\frac{1}{n}} \\
				&\Leftrightarrow \left(x_1x_2 \dots x_{n-1}\right)^{\frac{1}{n}} * \left(x_1x_2 \dots x_{n-1}\right)^{\frac{1}{n^2-n}} \\
				&\Leftrightarrow \left(x_1x_2 \dots x_{n-1}\right)^{\frac{1}{n} + \frac{1}{n^2-n}}\\
				&\Leftrightarrow \left(x_1x_2 \dots x_{n-1}\right)^{\frac{1}{n-1}} \\
				&\Leftrightarrow \sqrt[n-1]{x_1x_2 \dots x_{n-1}} = x_{n-1}\\
				\end{aligned}
			\end{equation}
			Damit gilt:  $\sqrt[n]{x_1x_2 \dots x_{n-1}*\sqrt[n-1]{x_1x_2 \dots x_{n-1}}} = \sqrt[n-1]{x_1x_2 \dots x_{n-1}}$ \\
			x_n einsetzen in Ungleichung  \\
			
			\begin{equation}
				\begin{aligned}
					\frac{x_1+x_2+ \dots +x_{n-1}+\sqrt[n-1]{x_1x_2 \dots x_{n-1}}}{n} &\geq \sqrt[n]{x_1x_2 \dots x_{n-1}*\sqrt[n-1]{x_1x_2 \dots x_{n-1}}} \quad |erstzen durch Term (s.o.)\notag \\
					\frac{x_1+x_2+ \dots +x_{n-1}+\sqrt[n-1]{x_1x_2 \dots x_{n-1}}}{n} &\geq  \sqrt[n-1]{x_1x_2 \dots x_{n-1}} \\
					\Leftrightarrow x_1+x_2+ \dots + x_{n-1}+\sqrt[n-1]{x_1x_2 \dots x_{n-1}} &\geq n*\sqrt[n-1]{x_1x_2 \dots x_{n-1}}	\\				
				    \Leftrightarrow x_1+x_2+ \dots + x_{n-1} &\geq n*\sqrt[n-1]{x_1x_2 \dots x_{n-1}} - \sqrt[n-1]{x_1x_2 \dots x_{n-1}} \\
				    \Leftrightarrow x_1+x_2+ \dots + x_{n-1} &\geq  \left(n-1\right)*	\sqrt[n-1]{x_1x_2 \dots x_{n-1}} \\
				    \Leftrightarrow \frac{x_1+x_2+ \dots + x_{n-1}}{n-1} &\geq \sqrt[n-1]{x_1x_2 \dots x_{n-1}} \Leftrightarrow P(n-1)\\				
				\end{aligned}
			\end{equation}
		\end{proof}
	\item 
	
  \end{itemize}

\begin{flushleft}
	(1.)
\end{flushleft}

\begin{itemize}
\item
	\begin{proof}
	 P(2) ist wahre Aussage \\
		$$x_1*x_2 \leq (\frac{x_1+x_2}{2})^2$ \\
		$$x_1*x_2 \leq \frac{(x_1+x_2)^2}{4}$  \\
		$$4*x_1*x_2 \leq (x_1+x_2)^2$ \\
		$$4*x_1*x_2 \leq {x_1}^2 + 2*x_1x_2 + {x_2}^2 | -4*x_1x_2$\\
		$$0 \leq {x_1}^2 - 2*x_1x_2 + {x_2}^2 $\\
		$$0 \leq  (x_1-x_2)^2 $ w.A., da $ x_1, x_2 \geq 0 $\\
		
	    
	\end{proof}
\end{itemize}

\newpage



\newpage
\section{Manipulation elementarer Funktionen}

Finden Sie Paare von \"aquivalenten Termen und formen Sie diese
schrittweise ineinander um. Geben Sie die verwendeten Regeln an.
\[
 \log_a\Bigl(n^{\log_b a}\Bigr), \sqrt[b]{\frac{a^n}{a^m}},
 b^{n \log a}, \log_b n, a^{\frac{n-m}{b}}, n(\log a + \log b),
 \log(a^nb^n),a^{(\log b^n)}.
\]
{\itshape Lösungen}
\begin{itemize}
\item $a \Leftrightarrow d$ \\
	  \[
      log_{a}(n^{log_{b}a}) \Leftrightarrow log_{b}a * log_{a}n \Leftrightarrow \frac{log_{a}a}{log_{a}b} * log_{a}n \Leftrightarrow \frac{1}{log_{a}b} * log_{a}n \Leftrightarrow \frac{log_{a}n}{log_{a}b} \Leftrightarrow log_{b}n
      \]
\item $b \Leftrightarrow e$ \\
	  \[
       \sqrt[b]{\frac{a^n}{a^m}} \Leftrightarrow \frac{\sqrt[b]{a^n}}{\sqrt[b]{a^m}} \Leftrightarrow \frac{a^{\frac{n}{b}}}{a^{\frac{m}{b}}} \Leftrightarrow a^{\frac{n-m}{b}}
      \]
\item $c \Leftrightarrow h$ \\
	  \[
        b^{n*log_{2}a} \Leftrightarrow (b^{log_{2}a})^n \Leftrightarrow (2^{log_{2}b * log_{2}a})^n \Leftrightarrow ((2^{log_{2}a})^{log_{2}b})^n \Leftrightarrow (a^{log_{2}b})^n \Leftrightarrow a^{n * log_{2}b}
      \]
\item $ f \Leftrightarrow g$
      \[
       n*(log(a) + log(b)) \Leftrightarrow n * log(a) + n* log(b) \Leftrightarrow log(a^n) + log(b^n) \Leftrightarrow log(a^n*b^n)
      \]
\end{itemize}



% /////////////////////// END DOKUMENT /////////////////////////
\end{document}
